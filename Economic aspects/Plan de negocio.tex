\documentclass[11pt]{article}
\usepackage[spanish]{babel}
\usepackage[utf8]{inputenx}
\usepackage{eurosym}
\usepackage{verbatim}
\usepackage{url}
\usepackage[pdftex]{graphicx}
\usepackage{hyperref}
\newcommand{\HRule}{\rule{\linewidth}{0.5mm}}

\begin{document}


\begin{titlepage}

\begin{center}


% Upper part of the page
\includegraphics[width=0.15\textwidth]{./logo}\\[1cm]    

\textsc{\LARGE Universidad Rey Juan Carlos}\\[1.5cm]
\textsc{\Large Máster Universitario en Software Libre}\\[0.2cm]
\textsc{Aspectos económicos del Software Libre}\\[0.2cm]

% Title
\HRule \\[0.4cm]
{ \huge \bfseries Plan de negocio - Software de gestión para autodiagnosticos en pacientes y médicos}\\[0.4cm]
\HRule \\[1.5cm]

% Author and supervisor
\begin{minipage}{1\textwidth}
\begin{flushleft} \large
\emph{Autor:}\\
Oliver \textsc{Mas}
\end{flushleft}
\end{minipage}

\vfill

% Bottom of the page
{\large \today}

\end{center}

\end{titlepage}



\tableofcontents
\newpage


\section{Introducción}
El negocio se basa en una serie de herramientas web y aplicación móvil para realizar diagnosticos y autodiagnosticos para detectar posibles problemas de salud y llevar un historial médico.
Se trata de un proyecto de Software Libre bajo licencia GPLv3. Con esta estrategia tratamos de abrir mercado dentro del software de gestión y diagnostico medicinal, ya que mucho software dedicado a este tipo de gestiones, están sujetas a patentes y estándares cerrados. El factor de Software Libre es vital cuando se trata de sanidad, pues con esta herramienta se intenta crear estándar abierto para la interoperabilidad entre hospitales o clínicas. Además, gracias a otras tecnologías hardware como dispositivos móviles con Android y Arduino, podrá abrirse puertas a crear protocolos o nuevos dispositivos económicos y fáciles de usar para que las personas se puedan diagnosticar. Por ejemplo, gracias a los sensores de gravedad y rotación de un dispositivo con Android, se podrían detectar problemas musculares sin que el paciente tenga la necesidad de desplazarse fisicamente a la consulta médica. También se podrá hacer una serie de ejercicios y progresiones para que el paciente tenga retroalimentación de su avance físico. 

Una fuente favorable de que este es un proyecto viable es que Android en su versión 4, gracias a la labor GSyC/Libresoft, incluye el protocolo ISO/IEEE 11073-20601. Una pila de comunucación por Bluetooth para comunicarse con dispositivos médicos, del que existe un proyecto en la forja de Morfeo llamado OpenHealth \url{http://openhealth.morfeo-project.org/}.

\newpage

\section{Análisis del modelo de negocio}
\subsection{Socios clave}
Socios claves son los hospitales o clínicas interesados en implantar este sistema. Para ayudar a la popularización de este producto, se puede lanzar aplicaciones móviles con funciones básicas para que la población se realicen autodiagnostico.

\subsection{Actividades clave}
La actividad clave de este modelo de negocio, es que se popularice entre los pacientes, y que sea recomendado por médicos. De esta manera el canal de distribución por el que se obtiene beneficios son los hospitales, que necesitarían implementar este software en sus instalaciones.

\subsection{Recursos clave}
Nuestros recursos claves son los canales de distribución, para que el proyecto avance adecuadamente se necesita que la gnete colabore en su uso y que llame la atención al principal aportador de ingresos que son hospitales y/o clínicas.
Claro que al ser un proyecto de Software Libre lleva consigo una serie de valores que lo hacen clave para crear una serie de estándares de documentación y tecnológicos para el mercado sanitario, que se mueve en el mercado de software propietario. Conseguiremos así abrir mercado a nuevos frabricantes especializados en el ámbito de la medicina.

\subsection{Propuestas de valor}
\begin{itemize}
\item Sencillez de tener su historial clínico siempre a mano tanto para pacientes como a médicos.
\item Poder realizar un autodiagnostico sencillo en su propio hogar.
\item Avisos de emergencia automáticos, asistencia personalizada.
\item Asistencia a distancia.
\item Alarmas personalizadas tanto para un paciente que necesita una revisioón personalizada, como un médico para atender a sus citas.
\item Evaluaciones de epidemias o análisis y gráficas de determinados problemas de salud, gracias a los datos demográficos obtenidos a dispositivos móviles.
\item Vademecum móvil y con posible capacidad de mejora pro cada médico.
\end{itemize}

\subsection{Canales de distribución}
Los canalaes de distribución pueden ser varios, desde una simple aplicación en mercado de aplicaciones del dispositivo móvil, una página web del hospital o incluso el mismo hospital será encargado de facilitar dispositivos para su autodiagnosticos.

\subsection{Segmentos de clientes}
\begin{itemize}
\item Pacientes de hospitales
\item Usuarios para autoevaluarse
\item Médicos
\end{itemize}

\subsection{Estructura de costes}
El coste más importante es realizar una primera versión del producto, en adelante se asume que los ingresos obtenidos podrán servir para las mejoras que cada Hospital/Clinica o pacientes requieran. En este caso hay que conseguir un Business Angel, hospital interesado, crowdfunding o que la comunidad se interese y aporte su tiempo voluntario para el desarrollo.
Dependiendo de que sistema de financiación inicial se consiga, el coste puede ser mínimo o más alto si se necesitara entregar el proyecto en un determinado plazo.
Suponiendo el caso de que el proyecto hay que entregarlo en un plazo de un año, contando con 5 desarrolladores cobrando 1500\euro mensuales y añadiendo gastos extraordinarios para desarrollo de diseños gráficos y hardware, saldría a un coste de 100000\euro.

\subsection{Fuente de ingresos}
Como usuarios finales, los pacientes no necesitarían pagar nada mas que a su seguro médico, o incluso nada, pues la aplicación estará de forma gratuita para autodiagnosticarse y conectarse con su centro médico.
Donde se obtendrían ingresos será por parte de los clínicas y/o hospitales que quieran implementar este software para sus pacientes como valor extra.

\section{Licencia}
Este obra está bajo una licencia de Creative Commons Reconocimiento-CompartirIgual 3.0 Unported. \url{http://creativecommons.org/licenses/by-sa/3.0/}.
The source code of this document can be found in:
	\begin{itemize}
		\item \url{git://gitorious.org/mswl_notes_2011_2012/mswl_notes_2011_2012.git}
		\item \url{https://git.gitorious.org/mswl_notes_2011_2012/mswl_notes_2011_2012.git}
	\end{itemize}
    \begin{center}
    \end{center}
\end{document}  
